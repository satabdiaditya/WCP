The contact map overlap problem, in its graph theoretical representation, is NP-hard. This implies that it is less probable that a polynomial time algorithm exists that computes the contact map overlap between two graphs of self-avoiding walks.

The structure of the proof is suggested in \citet{goip99}. The proof of the above is deduced with the help of two other lemmata. In the first step, it is proved that computation of the overlap of two $2$-stack graphs which have a maximum degree $3$ is NP-hard. The reduction is from the Planar $3$-SAT problem which is known to be NP-complete itself \citep{lich82}. Let the Planar $3$-SAT problem instance consist of $n$ variables $\{x_1,x_2,\cdots,x_n\}$ and $m$ clauses $\{c_1,c_2,\cdots,c_m\}$. This instance of the problem is mapped into two $2$-stack graphs $G_1$ and $G_2$. These two graphs contain variable and clause gadgets arranged in a linear layout according to a planar embedding of the formula. Between each variable and clause gadgets (nodes in the graph), enforcing gadgets are inserted which consist of $5m$ (on $10m$ nodes) edges. Once the two graphs are constructed, the constraints of the Planar $3$-SAT formula is satisfied by an order preserving bijection between the vertices of $G_1$ and $G_2$. The order-preserving bijection is as follows. The variable node in each variable gadget of $G_1$is mapped to the true literal node in the corresponding variable gadget of $G_2$. Further, nodes in the enforcing gadgets of $G_1$ are mapped to nodes in the corresponding gadget of $G_2$. Finally, the clause node in each clause gadget of $G_1$ is mapped to a true literal node in the corresponding clause gadget of $G_2$. The two graphs can thus be reduced to an instance of the Planar $3$-SAT problem and since the Planar $3$-SAT problem is NP-complete, it is NP-hard to compute the overlap of the two 2-stack graphs of maximum degree $3$.

The second step involves proving that it is also NP-hard to compute the overlap of two 2-stack graphs which have a maximum degree $1$. The proof follows similar argument, \emph{i.e.} the problem again is reduced to a Planar $3$-SAT problem in polynomial time. The only difference is the way in which the two graphs are constructed such that they are easily decomposable into two stacks each having degree $1$. The order-preserving bijection follows the same rules.

These two proofs are successfully used to construct the proof of NP-hardness of the computation of the contact overlap of two self-avoiding walks. The structure of the proof is very similar, the only difference being the enforcing gadgets used for the construction of the Planar $3$-SAT Problem. 