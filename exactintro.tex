\section{Introduction}
Though NP-hard problems are not solvable in polynomial time, they can be solved through exhaustive search. It is when the size of the search space grows, the running time becomes astronomically large or memory becomes a bottleneck. However, for some special cases, it is possible to design algorithms that are faster than the exhaustive search. These algorithms are called ``Exact" algorithms. There exists a number of such exact algorithms for the maximum CMO problem where Integer Linear programming is adopted. The approach taken by most of these algorithms is to formulate the contact map overlap problem as a maximization problem of linear function of integer variables subject to linear equalities and then solve these \emph{via} Branch and Bound type algorithms, thereby providing an upper bound to the optimal solution. When the gap between the upper bound and the optimal solution is the least, the solution is effective. The following two techniques are used to find the upper bound.
\begin{noindlist}
\item {\it Linear Programming:} This method for finding the bounds is more widely and successfully used by a number of applications \citep{anmy11}. In this technique, the variables in the linear function are allowed to have fractional values. For any integer linear programming problem, the efficiency of its solution depends on the choice of variables and constraints in its linear equality. So in order to tighten the LP bounds, it is often seen that another constraint is introduced which is known as {\it cuts}. Cuts are not used to eliminate any integer solution. It tightens the LP bounds by making the space for the fractional solutions smaller. In the ILP formulation of the Maximum CMO problem, it is seen that there are an exponential number of possible inequalities. Now, out of a number of mutually incompatible ways to align two residues, at most one of them can be chosen. There is an algorithm to deal with these exponential number of constraints. This is called the ``Separation" algorithm. Given a fractional solution, this algorithm finds a violated inequality if one exists.Once such an inequality is found, it becomes a cut in the LP formulation of the CMO problem. If the searching for such inequality takes polynomial time to solve, the exponential sized LP relaxation also takes polynomial time to find a solution. In case of the CMO problem, if $n_1$ and $n_2$ are the number of residues in the proteins that are to be aligned,then the time taken for the separation of these violated inequalities is $O(n_1\times n_2)$.

\item {\it Lagrangian Relaxation:} There are ILP problem formulations which consists of two sets of constraints - ``nice" constraints and ``bad" ones. When these bad constraints are removed the LP formulation becomes easily solvable. The strategy is to remove the bad constraints and add them to the objective function, each with a weight(or penalty incurred by a solution which does not satisfy the formulation). Once these bad constraints have been removed, the remaining Lagraingian problem becomes easily solvable and these constraints help us in obtaining an upper bound for the problem. In the case of the CMO problem, if $n_1$ and $n_2$ are the number of residues in the two proteins to be mapped, then the problem can be reduced to $O(n_1\times n_2)$ number of sequence alignment cases each being solvable in $O(n_1\times n_2)$ time using Dynamic Programming. So the ILP problem is reduced to generating the Lagrangian multipliers that yields the best upper bound. In most cases, determining these multipliers is similar to solving another instance of LP problem which is very time consuming. However, near optimal solutions  can be found using iterative procedures known as sub-gradient optimization. In this optimization problem, the Lagrangian problem is solved and the multipliers are updated depending on the solutions. Another use of the Lagrangian multiplier is to find near optimal solutions for a problem. Different heuristic procedures produce different solutions for different Lagrangian multipliers. These when embedded within an iterative procedure to define near optimal multipliers,the best solution produced over all iterations is considered the near optimal solution.
\end{noindlist}

The most powerful and efficient algorithms for finding exact solutions of contact map overlap is based on Integer Programming. This approach first formulates the CMO problem as the maximization of a linear function of some integer variable and then uses Lagrangian relaxation to solve the model and incorporates it into a Branch and bound search.



