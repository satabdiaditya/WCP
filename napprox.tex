\citep{agmw07} present the first result of computing the contact map overlap between two graphs in the case of $3$-dimensional self-avoiding walks. Let $G = (V,E)$ be any ordered graph with an ordered set of vertices $V = (v_1,v_2,\cdots,v_n)$ and $E$ be the edges arranged in increasing order of their left end-points \emph{i.e.} $E = \{e_1=(v_{s_1},v_{t_1}),\cdots,e_u=(v_{s_u},v_{t_u})\}$, where $s_1 \leq \cdots \leq s_u$ are the starting vertices of the edges. Further, let $\tau = (t_1,\cdots,t_u)$  and $\sigma$ be defined as the smallest integer $k$ such that $\tau$ is decomposed into $k$ monotonically increasing or decreasing subsequences. The decomposition of $\tau$ into $k$ monotonically increasing or decreasing subsequences takes place in $O(n \log n)$ time according to the algorithm suggested in \citet{k73}. Since $G$ is a constant graph of degree $n$, length of $\tau$ is $O(n)$ and the number of iterations to create this subsequence is at most $O(\sqrt n)$ \citep{erd35}.

Let the subsequence extracted at the $k^{\text{th}}$ iteration be given by $\tau _k = (t_{i1},....,t_{il})$ and the subgraph formed by these edges in $\tau _k$ be called $T_k$ where the edge set is given by $E _k =\{(s_{i1},t_{i1}),\cdots,(s_{il},t_{il})\}$. It can be shown that either $T_k$ is a stack (in a decreasing subsequence) or a queue (in an increasing subsequence). Two edges $e_i=(v_{s_i},v_{t_i})$ and $e_j=(v_{s_j},v_{t_j})$ are so taken that $t_i \in \tau _k$ at index $s_i$ and $t_j \in \tau _k$ at index $s_j$. Without loss of generality, for any subsequence one can consider $t_i < t_j$. If $\tau _k$ is an increasing subsequence, $s_i<s_j$ and consequently, $e_i$ does not contain $e_j$ and \emph{vice versa} for which $\tau _k$ becomes a queue. For a decreasing subsequence $s_i >s_j$ and $s_j$, the left index point of $e_j$, is less than $s_i$, the left index point of $e_i$, and $t_j$, the right index point of $e_j$, is more than $t_i$, the right index point of $e_i$. Thus $e_j$ contains $e_i$ and hence $T_k$ is a stack.

Now at any iteration,  $\tau _k$ is either a stack or a queue. Since each queue can be divided into two staircases, the total number of subgraphs is at most $O(\sqrt n)$. Again, since each subgraph is computed in $O(n \log n)$ time and there are $\sigma$ of such subgraphs, the total running time is $O(\sigma n \log n)$.

Analogous to the contact map overlap in  $2$-dimensional walks, to construct the overlap between two contact maps $G_1$ and $G_2$ of two $3$-dimensional walks, either $G_1$ or $G_2$ is decomposed into $\sigma$ stacks and staircases and then contact map overlap of each can be computed in polynomial time. Contact map overlap will be computed between these $\sigma$ subgraphs and $G_2$, achieving a $\sigma$-approximation algorithm. Summing up the running time of matching each of the smaller graphs to the other, the total running time is $O(n^3 \log n)$. 