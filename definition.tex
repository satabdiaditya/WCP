Let $Z$ be an integer grid in $\mathbb{R}^2$ or $\mathbb{R}^3$. A \emph{self-avoiding walk} is a one-to-one mapping $f:V =\{v_1,v_2,....,v_n\}\rightarrow Z$ such that $\|v_i - v_{(i+1)}\| = 1$ and $v_i \ne v_j$, $1 \leq (i,j) \leq n$. Self-avoiding walks are used to represent protein structures as physical evidences show that proteins cannot be represented by arbitrary edges.

A {\it contact map} is an ordered graph $G = (V,E)$ where $V = \{v_1,v_2,....,v_n\}$ is the set of vertices and $E = \{(v_i, v_j): \|j-i\| >1 \text{ and } \|v_i - v_j\| =1\}$ is the set of edges. Contact maps are useful representation of proteins where each vertex represents one amino acid and the edges are the proxy for the pairs of amino acids (not in the linear sequence) whose centroids are closer than a fixed threshold. Fig. \ref{fig:A self-avoiding walk} and \ref{fig:A contact Map Example} show the examples of a self-avoiding walk and its corresponding contact map.

Given two contact maps $G_1=(V_1,E_1)$ and $G_2=(V_2,E_2)$ where $V_1=\{v_1,v_2,\cdots,v_{n_1}\}$ and $V_2=\{u_1,u_2,\cdots,u_{n_2}\}$, $n_1$ and $n_2$ being the total number of vertices in $V_1$ and $V_2$ and two subsets $S_1 \subseteq \{v_1,v_2,\cdots,v_{n_1}\}$ and $S_2 \subseteq \{u_1,u_2,\cdots,u_{n_2}\}$ with $|S_1| = |S_2|$, the \emph{contact map overlap} is defined as the maximum cardinality of the set $|\{(v_i,v_j) \in E_1 : v_i,v_j \in S_1, (f(v_i), f(v_j)) \in E_2|$ under some {\it order-preserving bijection} $f$. Fig. \ref{fig:A contact Map overlap Example} explains a contact map overlap between the two graphs. 