\chapter{Critical Analysis}
\label{critique}
This chapter provides a critical analysis of comparing protein structures using CMO and also discusses possible extensions and research directions.

\section{Advantages of Using CMO}
\subsection{CMO vs RMSD}
Two most widely accepted scoring schemes for alignment of protein structures are : \emph{root mean square division} popularly known as RMSD \citep{kabsch76} and \emph{contact map overlap}(CMO). RMSD is a widely-used measure for quantifying the similarity between two protein structures. It is defined as the distance between two residues in comparing proteins depending on their relative position in space. So the matching process consists of two subtasks -  finding mapping of residue from one protein to another and also finding the best rotation and translation such that the RMSD is minimized under this mapping. From the definition, it can be said that although it is fairly easy to compute, it however treats protein structures as a rigid object. But in reality, proteins exhibit dynamic structural fluctuations; hence, keeping the structural aspect of the protein in mind, a need for a measure which takes the larger flexibility into account is needed.

Since contact maps concentrate on the structure of the protein and contains enough information to reconstruct the overall geometry of a protein's structure,they overcome the shortcoming of RMSD as a scoring mechanism. Again in a contact map, two residues are said to be in contact if the distance between them is less than a  given threshold. So it is insensitive to changes in distance between residues that are far apart from one another, allowing more flexibility than RMSD.

Overall, since RMSD is a global alignment process, any algorithm which uses RMSD as its scoring scheme, is a very poor indicator of the similarity between two proteins.CMO, on the other hand is robust, takes partial matching into account, is translation invariant and captures the intuitive notion of similarity very well \citep{agmw07}.

\subsection{CMO vs LCP Problem}
When it comes to protein structure alignment literature, two main mathematical frameworks known are the Maximum CMO problem and the LCP (Largest common point set) problem. In the LCP problem, one finds the maximum mapping of the largest cardinality where the RMSD of the matched residues is no more than a given threshold. The advantages of the LCP problem  are that it can solve the problem optimally and it is fairly easy to compute. However, despite solving the problem optimally, the running time complexity of the problem is so alarmingly high that it is practically impossible to compute the similarity between large proteins. Again since it uses RMSD as its scoring scheme, it does not include the flexibility of the protein structure in its calculations \citep{li13}. This is where CMO poses as an advantage. Since CMO is a graph-theoretic in nature, it captures the structural aspect of the protein thereby being a better process for aligning two proteins.

\section{Limitations and Areas of Improvement}
\subsection{CMO Problem vs Spectral Methods}
Quiet recently, researchers have started using spectral matching techniques of graphs to find the similarity between two protein 3D structures. The similarity measure between two proteins is a novel characterization technique called projections. \citet{bhat06} point out in his paper that an advantage spectral methods have over CMO is that comparison with spectral methods are based on two residues (one from each protein) unlike the CMO process which needs 4 residues(two from each protein). Also the spectral methods scale well with the change of distance parameter between two proteins \citep{bhat06}.
\vspace{-0.1cm}
\subsection{Improvement of Upper Bounds in Exact Algorithms}
Out of the five exact CMO approaches published, B \& Cut \citep{carr00}, Clique \citep{strick05}, LAGR \citep{cap04} and CMOS \citep{xie07} and the algorithm by \citet{anmy11} commonly known as \emph{A\_purva}, numerical results show that the last algorithm outperforms the remaining ones. The bounds derived by Lagrangian relaxation outperforms the other algorithms. It would be interesting to see if there could be reasonable improvement in the upper bounds via a combination of combinatorial approaches and mathematical programming approaches.
\vspace{-0.1cm}
\subsection{Practical Utility of Exact Algorithms}
Most of the experiments performed on protein structure similarity are done on structures which have been biologically proved to be similar. Aligning proteins with different structures still are much harder than those with similar structure. In experiments performed by \citet{anmy11} it is seen that while the time for aligning similar structures(domains from the same family) varies from $0.02$ seconds and $2.14$ seconds, the time taken to align dissimilar structures(domains from different families) is from $3.47$ seconds to more than $1800$ seconds. So the practicality of such exact algorithms are questionable. Experiments in this domain are mostly performed on small to medium sized proteins(e.g. 60 residues) and they have consumed two hours of CPU time on a standard workstation \citep{xie07}. It would be interesting to see how far these algorithms can be extended to incorporate
\vspace{-0.1cm}
\subsection{Biological Information for 3D-optimal Alignment}
One of the major drawbacks in this field of measuring similarity between 3D protein structures is the absence of easily available biological information \citep{xie07}. This is an important research direction because many instances, although mathematically important for testing the algorithm are not worth aligning from a biological perspective. This fact is reiterated in certain cases where the algorithm has given near optimal solutions but the secondary structure information of them show that while one protein consisted of all helixes, the other had strands. It would be interesting to see if secondary structure information could be incorporated in the construction of algorithms that compute contact map overlap.
\vspace{-0.1cm}
\subsection{Machine Learning for CMO Problem}
Machine learning being one of the most recent interesting research domains, one can also solve the maximum CMO problems using concepts like structure prediction. A relevant literature is available over more recent set of publications \citep{ksck13,hozr13,zhhz05,wewm05,terp10,vupf08}. Additionally, the discrete optimization problem in maximum CMO can be replaced by a suitable sub-modular optimization problem \citep{iybi12,iybi13,iyjb13} which can possibly lead to better run time complexity and bound.

