\begin{abstract}
A fundamental problem in molecular biology, with applications in domains like medicine, biotechnology, computational biology and chemistry, is to study the three-dimensional structure of a protein as the functional characteristics of a protein are attributed to its structure. Proteins that are similar in structure tend to share similar properties; hence finding similarity measure between two given proteins has garnered sufficient interest. Out of the several measures proposed, Contact Map Overlap (CMO) is one of the most reliable and robust measures of protein structure similarity. The measure is maximized when the amino acid residues of two proteins are aligned to maximize the number of common residue contacts. Additionally, such measure facilitates the grouping of proteins into functionally similar clusters and can potentially incorporate experts' feedback in the modeling process. Quite justifiably, significant research effort has been invested over the last two decades in the maximum CMO problem and this report, in its limited page budget, presents a brief summary of some of the most important works that have shaped the course of the research in this problem.

To begin with, the report provides a formal description of the maximum CMO problem and illustrates that the calculation of the overlap of contact maps between two proteins is an NP-hard problem. Then the approximation algorithms for both $2$ and $3$-dimensional structures are explored which is followed by a succinct description of an exact algorithm based on integer programming that provably performs better than other existing algorithms for maximum CMO problem in $2$-dimensional structure. Finally, a critical analysis of the algorithms discussed is appended and possible extensions and research directions are previewed.
\end{abstract}

